\documentclass[11pt]{article}
\usepackage[backref, colorlinks=true, citecolor=red, urlcolor=blue]{hyperref}

\newcommand{\handout}[5]{
  \noindent
  \begin{center}
  \framebox{
    \vbox{
      \hbox to 5.0in { {\bf } \hfill #2 }
      \vspace{4mm}
      \hbox to 5.0in { {\Large \hfill #5  \hfill} }
      \vspace{2mm}
      \hbox to 5.0in { {\em #3 \hfill #4} }
    }
  }
  \end{center}
  \vspace*{4mm}
}

\newcommand{\lecture}[4]{\handout{#1}{#2}{#3}{#4}{#1}}

\parindent 0in
\parskip 1.5ex

\begin{document}

\lecture{Midterm Exam Report}{Fall 2017}{SRIVATS BHARADWAJ}{Advance Algorithm Programming}

\section{Summary of the two methods}

\subsection{Hedcutter method}
Algorithms for computing Voronoi diagram:
An ordinary Voronoi diagram is formed by a set of points in the plane called the generators or generating points. Every point in the plane is identified with the generator which is closest to it by some metric. The common choice is to use the Euclidean L2 distance metric. 
                
The algorithm draws a set of right cones with their apexes at each generator. The cones all have the same height and are viewed from above the apexes with an orthogonal projection. In addition, each cone is given a unique colour which acts as the generator’s identity. Since the cones must intersect if there is more than one generator, the z-buffer determines for each pixel which cone is closer to the viewer and assigns that pixel the appropriate colour value. We can then scan the resulting image and determine which generator is closest to each pixel by using the unique colours. This technique allows us to compute discrete Voronoi diagrams extremely quickly and perform computations on the resulting regions. 

\subsection{Voronoi method}
A centroidal Voronoi diagram has the interesting property that each generating point lies exactly on the centroid of its Voronoi region. The centroid of a region is defined as

where A is the region, x is the position and r (x) is the density function. For a region of constant density r , the centroid can be considered as the centre of mass.
A centroidal Voronoi diagram is a minimum-energy configuration in the sense that it minimizes.

Stippling with Weighted CVDs
The centroidal Voronoi diagrams incorporate the idea of a density function r (x; y) which weights the centroid calculation. Regions with higher values of r will pack generating points closer than regions with lower values. During the iteration of Algorithm 1, the darker regions of the image appear to “attract” more points. We can use Algorithm 1 directly to generate high-quality stippling images by treating a grayscale image as a discrete two dimensional function f (x; y) where x; y ϵ [0; 1] and 0 ≤ f (x; y) ≤ 1
is the range from a black pixel to a white pixel. Define a density function ρ(x, y) = 1- f (x, y). 



\section{Comparison of the two methods}

Q1. Do you get the same results by running the same program on the same image multiple times?\\
A1. No. I got  slightly different results even though the command arguments were the same.
Because the starting points in the calcu\-lation of the Voronoi regions randomly selected.

Q2. If you vary the number of the disks in the output images,  do these implementations produce
the same distribution in the final image? If not, why?\\
A2. Both implementations produce different distribution in the final image because theys use
various convergence criteria.

Q3. If you vary the number of the disks in the output images,  is a method faster than the other? \\
A3. {\it voronoi} method is faster.

Q4. Does the size (number of pixels), image brightness or contrast of image increase or decrease their difference? \\
A4. The number of pixels, image brightness and contrast of image increase the difference between methods.

Q5. Does the type of image (human vs. machine,  natural vs. urban landscapes, photo vs. painting, etc) increase or
decrease their difference? \\
A5. I did not notice the difference between methods depending on the type of image.

Q6. Are the outputs of these stippling methods different  the hedcut images created by artists (e.g. those from the
\href{http://www.wsj.com/articles/SB10001424052748704207504575129961786135180}{Wall Street Journal})? \\
A6. In the images generated by the program, the regular structure of Voronoi regions is clearly visible.

\section{Improvement of hedcuter method}

\end{document}
